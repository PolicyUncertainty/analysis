\begin{table}[htbp]
    \centering
    \caption{Retirement Decisions Under Different Information and Reform Scenarios}
    \label{tab:retirement_scenarios}
    \begin{threeparttable}
    \begin{tabular}{lcccc}
        \toprule
        \multirow{2}{*}{\textbf{Outcome}} & \multicolumn{2}{c}{\textbf{Expected Reform}} & \multicolumn{2}{c}{\textbf{No Expected Reform}} \\
        \cmidrule(lr){2-3} \cmidrule(lr){4-5}
         & \textbf{Uninformed} & \textbf{Informed} & \textbf{Uninformed} & \textbf{Informed} \\
        \midrule
        Expected Retirement Age & 64.95 & 62.72 & 64.16 & 62.65 \\ \addlinespace
        Expected Lifetime Income (Tsd.) & 956.00 & 946.11 & 951.97 & 947.07 \\
        Private Wealth at Retirement (Tsd.) & 202.06 & 200.33 & 201.28 & 195.94 \\
        Pension Wealth at Retirement (Tsd.) & 53.45 & 79.38 & 59.97 & 82.02 \\ \addlinespace
        Full-Time Work Probability & 1817.82 & 1811.73 & 1820.17 & 1812.68 \\
        Consumption (Tsd.) & 29.89 & 30.01 & 29.92 & 30.11 \\
        \bottomrule
    \end{tabular}
    \begin{tablenotes}
        \small
        \item \textit{Notes:} This table presents expected outcomes for different combinations of 
        information states and reform expectations. Informed individuals know their health status, while 
        uninformed individuals face uncertainty. Monetary values are in thousands.
    \end{tablenotes}
    \end{threeparttable}
\end{table}